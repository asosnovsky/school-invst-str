\documentclass[12pt]{article}

\include{amsfonts,amssymb,amsmath,amsthm,txfonts,pxfonts,mathrsfs,enumitem}
\usepackage[headheight=15pt,paperwidth=8.5in, paperheight=11in]{geometry}
\usepackage{lastpage}
\usepackage{fancyhdr}


\newcounter{teamnumb}

\setcounter{teamnumb}{54301}


\ttfamily{
\lhead{Team\ \# \theteamnumb}\rhead{Page \thepage\ of \pageref{LastPage}}}
\cfoot{}

\renewcommand{\headrulewidth}{0pt}

\pagestyle{fancy}

\title{GG NO RE}
\begin{document}
\maketitle

\setcounter{page}{1}
\section*{Summary}
\thispagestyle{empty}
First we created a scale to benchmark the top 20 schools by incorporate PR. After we finding school that are smaller in scale both by grant and the number of student itself that share the same characteristic of the same top school in benchmark. \\

school that are close to the same benchmark schools are considered to be in the same class. The way we do that is by using the kNN algorithm to determine the level of closeness (classification). \\ 

For each class of school, we identify the school that received the smallest in grants and size. We then determine the maximum investment for that school per student by: \\
 $g_j^{[i]} = \frac{G_o^{[i]}}{•\#(U_o^{[i]})} - \frac{G_j^{[i]}}{•\#(U_j^{[i]})}$ \\ 
 
 
 




\section{Introduction}


\clearpage
\section{Assumptions and Rationale/Justification}

\clearpage
\section{Model Design and Justification} 
\subsection{Notation}
\begin{itemize}
\item $U_j^{[i]} = $ university set of class $i$
\item $U_o^{[i]} = $ benchmark university set of class $j$

\item $U_j^{[i]} = \{RR, UR, PR, \{T_j\}, \{P_j\}, GR, G   \} $


\end{itemize}
RR = Retention rate\\
UR = Umemployment rrate\\
PR = poverty rate\\
Tj = Tuition for j class of student (rich poor)\\
Pj = Population of student in j class\\
GR = grad rate\\
G  = grant\\


\section{Model Testing and Sensitivity Analysis} 

\section{Conclusions}
\clearpage

\begin{thebibliography}{10}
	\bibitem{Paolo} Brandimarte, Paolo. ``Stock Portfolio Optimization."  \emph{Numerical Methods in Finance and Economics: A MATLAB-based Introduction}. 2nd ed. Hoboken, N.J.: Wiley Interscience, 2006. 65 - 81, 571. Print.

	\bibitem{C} Corliss, George. ``Which Root Does the Bisection Algorithm Find?" \emph{SIAM Rev. SIAM Review} 19.2 (1977): 325-27. Print.
	
	\bibitem{L} Lummer, Scott. ``Taming Your Optimizer: A Guide Through the Pitfalls of Mean-Variance Optimization." \emph{Global Asset Allocation: Techniques for Optimizing Portfolio Management}. Ed. Jess Lederman. Illustrated ed. Vol. 29. New York: Wiley, 1994. Print.
	
	\bibitem{Robert} Michaud, Richard O., and Robert Michaud. ``Estimation Error and Portfolio Optimization: A Resampling Solution." \emph{Journal Of Investment Management} 6.1 (2008): 8-28. Print.
	
	\bibitem{Pat2} Pat. ``Alpha Alignment." \emph{Portfolio Probe}. 9 July 2012. Web. 13 Oct. 2015.
	
	\bibitem{Pat} Pat. ``The Top 7 Portfolio Optimization Problems."  \emph{Portfolio Probe}. 5 Jan. 2012. Web. 4 Dec. 2015.

	\bibitem{Jay} Walters, Jay. ``What Is the Black-Litterman Model." \emph{BlackLitterman.org}. 2013. Web. 20 Nov. 2015.

	\bibitem{Wayne} Wayne, Thorp. ``Mean Variance Optimization."  \emph{Computerized Investing}. Web. 10 Oct. 2015.
	
	\bibitem{Winston} Winston, Wayne L. ``Nonlinear Programming."  \emph{Operations Research: Applications and Algorithms}. 4th ed. Belmont, Calif.: Duxbury, 2003. Print.
\end{thebibliography}

\end{document}
